\cleardoublepage
\section*{Zusammenfassung}

Seit dem Beginn der Raumfahrt spielt Software eine kritische Rolle in jeder gestarteten Mission. Die Komplexität, Autonomie und Funktionalität stieg sowohl bei bemannten als auch unbemannten Missionen über die Jahre immer weiter an, was zu einem exponentiellen Wachstum bei der Software führte.
Obwohl großer Aufwand betrieben wird um die Zuverlässigkeit von Software sicherzustellen, gibt es immer wieder Fehler die zum scheitern einer Mission führen. Gleichzeitg können neuartige Lösungen für solche Probleme nicht in großem Maßstab getestet werden, da die Raumfahrtindustrie die damit verbundenen Risiken scheut.

CubeSats, kleine, standardisierte Satelliten, ermöglichen es diese Hürde zu überwinden. Die relative geringen Kosten erlauben es mit neuen Technologien zu experimentieren.

Um die wachsende Softwarekomplexität besser handhaben zu können bieten sich Programmiersprachen mit höherem Abstraktionslevel, wie etwa Python, an. Python ist eine einfache und schnell zu erlernende Sprache, die durch diese einfachere Nutzbarkeit hilft, mit Komplexität umzugehen. Die Standardimplementierung von Python benötigt funktioniert jedoch nicht auf der typischerweise in der Raumfahrt eingesetzten Hardware. Hier schafft MicroPython Abhilfe, eine Implementierung von Python für Geräte mit beschränktem Speicher.

Diese Arbeit evaluiert ob MicroPython auf CubeSats benutzt werden kann um die Softwarekomplexität besser beherrschbar zu machen. Die Sprache wird anhand von Beispielen aus der Entwicklung des MOVE-II Projektes analysiert. MicroPython wird jedoch nicht auf dem Flugmodel zum Einsatz kommen, da diese Evaluierung zu spät im Designprozess des Satelliten passiert.

To evaluate the space readiness of MicroPython a number of methods were used. First, a programming language evaluation using a canonical set of criteria showed the language to be well designed and having good readability, which is important for software maintenance. A project-based evaluation was done to investigate the suitability of the language for an example software taken from the MOVE-II project. This showed that while Python's usability makes it attractive, but the execution speed and storage requirements were problematic. MicroPython addresses the storage problem, but not the speed problem. After that, example implementations of software needed by MOVE-II were done to investigate whether or not programs written with MicroPython can fulfill all requirements in practice. The resulting programs complexity was compared between using MicroPython and C. It was possible to implement all needed functionality and the complexity was generally lower when using MicroPython. However, the benefits were stronger for programs that provide high level control of the system, and weaker for programs that interact directly with the hardware, like device drivers. To address the slower speed of the MicroPython programs, the creation of modules written in C but used from Python was investigated. MicroPython allows doing so in a straightforward fashion and the resulting speed was in the same order of magnitude than the C version.

The results of this evaluation are promising, but further work is needed. The evaluation method needs refinement to be more generally applicable and better specified. To prove MicroPython's space readiness, ultimately it has to be shown working in space. This could be done on the MOVE-II CubeSat after it's primary mission is fulfilled, by deploying MicroPython and a test program as a software update. While MicroPython can be used for all types of programs, it is best suited for writing the top level control software, implementing the systems operations. Lower level software, like drivers or performance critical algorithms can be prototyped in MicroPython, but are better written as C modules usable from Python.

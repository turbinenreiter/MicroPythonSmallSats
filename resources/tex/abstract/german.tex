\cleardoublepage
\section*{Zusammenfassung}

Seit dem Beginn der Raumfahrt spielt Software eine kritische Rolle in jeder gestarteten Mission. Die Komplexität, Autonomie und Funktionalität stieg sowohl bei bemannten als auch unbemannten Missionen über die Jahre immer weiter an, was zu einem exponentiellen Wachstum bei den geschriebenen Softwarezeilen führt.
Obwohl großer Aufwand betrieben wird um die Zuverlässigkeit von Software sicherzustellen, gibt es immer wieder Fehler die zum scheitern einer Mission führen. Gleichzeitg können neuartige Lösungen für solche Probleme nicht in großem Maßstab getestet werden, da die Raumfahrtindustrie die damit verbundenen Risiken scheut.
Um in diesem Spannungsfeld eine Weiternetwicklung zu ermöglichen, untersucht diese Arbeit das Potential von MicroPython, einer Implementierung der Programmiersprache Python für eingebette Systeme, für die Anwendung auf CubeSats. Dazu wird die Sprache, Impelemtierung und Werkzeuge am praktischen Beispiel des MOVE-II CubeSat Projektes getestet.

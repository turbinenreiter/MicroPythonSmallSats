\cleardoublepage
\section*{Zusammenfassung}

Seit dem Beginn der Raumfahrt spielt Software eine kritische Rolle in jeder gestarteten Mission. Die Komplexität, Autonomie und Funktionalität stieg sowohl bei bemannten als auch unbemannten Missionen über die Jahre immer weiter an, was zu einem exponentiellen Wachstum bei der Software führte.
Obwohl großer Aufwand betrieben wird um die Zuverlässigkeit von Software sicherzustellen, gibt es immer wieder Fehler die zum scheitern einer Mission führen. Gleichzeitig können neuartige Lösungen für solche Probleme nicht in großem Maßstab getestet werden, da die Raumfahrtindustrie die damit verbundenen Risiken scheut.

CubeSats, kleine, standardisierte Satelliten, ermöglichen es diese Hürde zu überwinden. Die relative geringen Kosten erlauben es mit neuen Technologien zu experimentieren.

Um die wachsende Softwarekomplexität besser handhaben zu können bieten sich Programmiersprachen mit höherem Abstraktionslevel, wie etwa Python, an. Python ist eine einfache und schnell zu erlernende Sprache, die durch diese einfachere Nutzbarkeit hilft, mit Komplexität umzugehen. Die Standardimplementierung von Python benötigt funktioniert jedoch nicht auf der typischerweise in der Raumfahrt eingesetzten Hardware. Hier schafft MicroPython Abhilfe, eine Implementierung von Python für Geräte mit beschränktem Speicher.

Diese Arbeit evaluiert ob MicroPython auf CubeSats benutzt werden kann um die Softwarekomplexität besser beherrschbar zu machen. Die Sprache wird anhand von Beispielen aus der Entwicklung des MOVE-II Projektes analysiert. MicroPython wird jedoch nicht auf dem Flugmodel zum Einsatz kommen, da diese Evaluierung zu spät im Designprozess des Satelliten passiert.

Um die Eignung von MicroPython für Raumfahrtanwendungen fest zu stellen wurden verschieden Methoden angewandt. Zuerst zeigte ein Evaluierung der Sprache anhand klassischer Evaluationskriterien für Programmiersprachen ihre hohe Lesbarkeit auf, was besonders für die Softwarewartung eine große Rolle spielt. Mittels einer projektbasierten Evaluation wurde die Anwendbarkeit der Sprache auf eine Beispielsoftware aus dem MOVE-II Projekt überprüft. Hier zeigte Python Stärken im Bereich Nutzbarkeit, aber auch Schwächen in den Bereichen Ausführungsgeschwindigkeit und Speicherverbrauch. MicroPython hilft das Speicherproblem zu verringern, nicht aber das Problem der langsameren Ausführungsgeschwindigkeit. Danach wurden Softwarebeispiele aus dem MOVE-II Projekt in MicroPython implementiert um zu überprüfen ob so erstellte Software alle daran gestellten Anforderungen erfüllen kann. Dabei wurde die Komplexität des Programmcodes zwischen den Implementierungen in Python und in C verglichen. Alle Programme konnten erfolgreich erstellt werden und die Komplexität des Python Programmcodes war generell niedriger als die des C Programmcodes. Die größten Vorteile konnten dabei jedoch bei den Teilen der Software erzielte werden, die sich mit der abstrakten Kontrolle des Systems befassten, während hardwarenahe Teile, wie etwa Gerätetreiber, weniger Komplexitätsreduktion erfuhren. Um die geringere Ausführungsgeschwindigkeit zu verbessern ist es mögliche Module in C zu programmieren, die dann in Python als Bibliotheken verwendet werden können. MicroPython ermöglicht dies in einfacher Art und Weise und die so erreichten Geschwindigkeiten waren im selben Größenordnungsbereich wie die der reinen C Versionen.

Die Ergebnisse dieser Auswertung sind vielversprechend, aber weitere Untersuchungen sind nötig. Die Evaluationsmethoden sollten verfeinert und generalisiert werden. Dazu ist auch eine formalere Spezifikation nötig. Um MicroPythons Anwendbarkeit für die Raumfahrt tatsächlich zu beweisen, muss ein Experiment im Orbit durchgeführt werden. Das könnte zum Beispiel auf dem MOVE-II CubeSat passieren, nachdem dieser seine Primärmission erfüllt hat, indem MicroPython und ein Testprogramm als Softwareupdate aufgespielt wird.

Obwohl MicroPython für alle Programmtypen genutzt werden kann, machen die Stärken und Schwächen der Sprach den Bereich der operationalen Kontrollsoftware, welche die Funktionen des Systems steuern, das geeignetste Ziel für ihren Einsatz. Für hardwarenahe Programme, wie Gerätetreiber oder laufzeitkritische Algorithmen können mit MicroPython funktionale Prototypen erstellt werden, das Entwickeln von C Modulen die als Python Bibliotheken nutzbar sind ist hier aber sinnvoll.

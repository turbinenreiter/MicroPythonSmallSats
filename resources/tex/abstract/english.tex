\cleardoublepage
\section*{Abstract}

Since the beginning of the space age, software has always been a critical aspect for any space mission launched. Over the decades, more complexity, autonomy and functionality was added to both unmanned and manned missions, yielding in an exponential growth of the lines of codes used in space projects over the years. Despite a lot of effort being put into ensuring reliable software on those missions, some of them failed. Still, as the space industry is a risk-averse business, testing of novel approaches in space programs cannot be done on a large scale. To overcome this limitation, this thseis investigates the potential use of MicroPython, an implementation of Python for constrained systems, for use on CubeSats by analyzing the language and tools in practical examples from the MOVE-II CubeSat project.
